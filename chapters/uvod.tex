\chapter{Úvod}
Koža je najväčší orgán ľudského teľa, ktorý pokrýva telo človeka. Kvôli tomu je prirodzené, že podlieha rôznym chorobám a je náchylná na početné zranenia. Medzi časté zranenia patria aj kožné rany, ktoré sú spôsobené rôznymi faktormi anatomického a fyziologického charakteru, či už sa jedná vnútorne príčiny, alebo vonkajšie. Rany môžu byť od tých menej vážnych, až po tie najvážnejšie. Menej vážne rany zasahujú väčšinou do pokožky, zamše a podkožného tuku. Avšak vážnejšie rany už môžu zasahovať aj nervovo-cievnych zväzkov a orgánov, a tým tak významne ohrozovať život postihnutého pacienta. Jedno z najvýznamnejších delení rán je delenie podľa priebehu. Takéto delenie rozdeľuje rany na, rany akútne a rany chronické. Akútne rany vznikajú v zdravom kožnom tkanive, hoja sa v krátkom čase a bez komplikácií. S takýmito ranami sa človek stretá často vo svojom živote, či už pri porezaní kože, odretí kože, alebo iných bežných vonkajších kožných poraneniach. Chronické rany, ktoré sú stredobodom tejto práce, na druhú stranu trvajú dlhšie než 4 týždne, a aj napriek všetkej odpovedajúcej liečbe nevykazujú tendenciu hojenia. Za najbežnejšiu príčinu takýchto rán sa považujú lokálne poruchy výživy kože, pôsobenie tlaku, alebo systémové ochorenie. 

Táto diplomová práca vznikla z dôvodu neexistencie nástroja, ktorý by pomohol lekárom a zdravotným sestrám so sledovaním chronických rán, ich priebehu a vývoja v čase. Pomocou aplikácie by malo byť možné detekovať, lokalizovať a určiť plochu chronickej rany. Zároveň by malo byť možné porovnávať jednotlivé výsledky danej rany v histórií a tak adekvátne reagovať na danú situáciu zmenou liečby, poprípade zavedením opatrení proti zhoršovaniu diagnózy. V momentálnej situácií určenie plochy chronickej rany prebieha zdravotnými sestrami za pomoci pravítok a iných merných pomôcok. Táto klasická metóda získavania informácií o rane a zisťovania plochy nie je veľmi efektívna a ani presná, keďže rany môžu mať rôzne tvary a podoby.
	
Nasledujúca kapitola sa venuje teoretickému rozboru a obsahuje informácie potrebné k základnému pochopeniu problematiky chronických rán. Ďalšia kapitola sa venuje analýze a návrhu samotnej mobilnej aplikácie a jej serverových súčastí. Konkrétne popisuje návrh grafického užívateľského rozhrania, ale aj návrh aplikačného rozhrania REST a výber vhodných technologických prostriedkov. Hneď po tom nasleduje kapitola, ktorá rozoberá technológie, ktoré boli potrebné pri vývoji aplikácie a jej súčastí.