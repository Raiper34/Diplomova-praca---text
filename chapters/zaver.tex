\chapter{Záver}
Cieľom diplomovej práce bolo vytvorenie aplikácie pre zariadenia s operačným systémom Android a zariadenia s operačným systémom Windows pre detekciu, lokalizáciu a určenie plochy chronických rán. Táto aplikácia má slúžiť zdravotným sestrám, doktorom a ošetrovateľom ako pomoc pri vyhodnocovaní a sledovaní chronickej rany počas trvania liečby. Aplikácia je postavená na programovacom jazyku Typescript, hybridnom aplikačnom rámci Ionic a Electron. Detekcia chronickej rany prebieha na strane servera, kde sa využíva programovací jazyk Python, pre RESTful aplikačné rozhranie aplikačný rámec Flask a pre spracovanie obrazu knižnica OpenCV. Dáta sa ukladajú do NoSQL databáze MongoDB.

Práca na začiatku začínala kapitolou \ref{chap:teoreticky-rozbor}, ktorá sa venovala teoretickému rozboru a obsahuje informácie potrebné k základnému pochopeniu problematiky chronických rán. Ďalšia kapitola \ref{chap:analyza-navrh-riesenia} sa zaoberala analýzou a návrhom samotnej mobilnej aplikácie a jej serverových súčastí. Konkrétne popisovala návrh grafického užívateľského rozhrania, ale aj návrh aplikačného rozhrania REST a výber vhodných technologických prostriedkov. Hneď potom nasleduje kapitola \ref{chap:pouzite-techologie}, ktorá rozoberala technológie, ktoré boli potrebné pri vývoji aplikácie a jej súčastí. V poradí ďalšou a určite najvýznamnejšou kapitolou je kapitola \ref{chap:implementacia}, ktorá pojednávala o samotnej implementácií. Celá práca je ukončená kapitolou \ref{chap:testovanie-vyhodnocovanie} o testovaní a vyhodnocovaní implementovaného riešenia a sú navrhnuté ďalšie kroky a smerovanie práce.

Finálna Android a Windows aplikácia, ktorá vznikla v rámci tejto diplomovej práce dokáže detekovať, lokalizovať a určiť plochu zosnímanej chronickej rany. Aplikácia detekuje tieto rany poloautomaticky výberom bodu vo vnútri rany za pomoci Region Growing algoritmu nad RYKW pravdepodobnostnou mapou, alebo v prípade zložitejších chronických rán, kedy poloautomatická detekcia výrazne zlyháva je možné použiť manuálnu detekciu pomocou ohraničenia plochy rany pomocou prsta, alebo myši. Výpočet reálnej plochy rany je potom zabezpečený pomocou užívateľom zvolenej mierky, kedy sa vždy jeden pixel plochy prevádza do nejakej miery reálneho sveta. Aplikácia je teda použiteľná do určitej miery v klinickej praxi, avšak mala by byť ešte ďalej testovaná a vylepšovaná, poprípade by sa mala užšie zamerať na niektoré špecifické typy rán v konkrétnom štádií. Aplikácia ďalej môže byť ešte rozšírená tak, aby bola schopná detekovať rôzne iné defekty kože. Takýmito príkladmi môžu byť napríklad hematómy, alebo materské znamienka.