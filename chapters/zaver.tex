\chapter{Záver}
Cieľom diplomovej práce je vytvorenie aplikácie pre zariadenia s operačným systémom Android a zariadenia s operačným systémom Windows pre detekciu, lokalizáciu a určenie plochy chronických rán. Táto aplikácia má slúžiť zdravotným sestrám, doktorom a ošetrovateľom ako pomoc pri vyhodnocovaní a sledovaní chronickej rany počas trvania liečby. Aplikácia bude postavená na programovacom jazyku Typescript, hybridnom aplikačnom rámci Ionic a Electron. Vyhodnocovanie chronickej rany bude prebiehať na strane servera, kde sa bude využívať programovací jazyk Python, pre RESTful aplikačné rozhranie aplikačný rámec Flask a pre spracovanie obrazu knižnica OpenCV. Dáta sa budú ukladať do NoSQL databáze MongoDB.

Cieľom tohoto semestrálneho projektu bolo preštudovať literatúru týkajúcu sa chronických rán, aplikačného rámca Ionic a knižnice pre spracovanie obrazu OpenCV. Zo štúdia týchto literárnych zdrojov bola vytvorená predovšetkým kapitola Teoretický rozbor a kapitola Prehľad použitých technológií. Ďalej mal byť vykonaný návrh aplikácie pre lokalizáciu, detekciu a určenie plochy chronickej rany. Návrhom sa zaoberá kapitola Analýza a návrh riešenia, v ktorej sú pomerne podrobne navrhnuté jednotlivé časti budúcej programovej implementácie. 

Na semestrálnom projekte sa bude naďalej pracovať za účelom vytvorenia kvalitnej diplomovej práce. Ďalším krokom v bude implementácia a následné testovanie aplikácie pre prostredie Android a Windows. Okrem toho budú zhodnotené dosiahnuté výsledky a budú navrhnuté možnosti pokračovania projektu.