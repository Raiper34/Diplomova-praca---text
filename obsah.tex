%=========================================================================
% (c) Michal Bidlo, Bohuslav Křena, 2008

\chapter{Úvod}
Abychom mohli napsat odborný text jasně a~srozumitelně, musíme splnit několik základních předpokladů:
\begin{itemize}
\item Musíme mít co říci,
\item musíme vědět, komu to chceme říci,
\item musíme si dokonale promyslet obsah,
\item musíme psát strukturovaně. 
\end{itemize}

Tyto a další pokyny jsou dostupné též na školních internetových stránkách \cite{fitWeb}.

Přehled základů typografie a tvorby dokumentů s využitím systému \LaTeX je 
uveden v~\cite{Rybicka}.

\section{Musíme mít co říci}
Dalším důležitým předpokladem dobrého psaní je {\it psát pro někoho}. Píšeme-li si poznámky sami pro sebe, píšeme je jinak než výzkumnou zprávu, článek, diplomovou práci, knihu nebo dopis. Podle předpokládaného čtenáře se rozhodneme pro způsob psaní, rozsah informace a~míru detailů.

\section{Musíme vědět, komu to chceme říci}
Dalším důležitým předpokladem dobrého psaní je psát pro někoho. Píšeme-li si poznámky sami pro sebe, píšeme je jinak než výzkumnou zprávu, článek, diplomovou práci, knihu nebo dopis. Podle předpokládaného čtenáře se rozhodneme pro způsob psaní, rozsah informace a~míru detailů.
