% Tento soubor nahraďte vlastním souborem s přílohami (nadpisy níže jsou pouze pro příklad)
% This file should be replaced with your file with an appendices (headings below are examples only)

% Umístění obsahu paměťového média do příloh je vhodné konzultovat s vedoucím
% Placing of table of contents of the memory media here should be consulted with a supervisor
%\chapter{Obsah přiloženého paměťového média}

%\chapter{Manuál}

%\chapter{Konfigurační soubor} % Configuration file

%\chapter{RelaxNG Schéma konfiguračního souboru} % Scheme of RelaxNG configuration file

%\chapter{Plakát} % poster

\chapter{Jak pracovat s touto šablonou}
\label{jak}

V této kapitole je uveden popis jednotlivých částí šablony, po kterém následuje stručný návod, jak s touto šablonou pracovat. 

Jedná se o přechodnou verzi šablony. Nová verze bude zveřejněna do~konce roku 2017 a~bude navíc obsahovat nové pokyny ke správnému využití šablony, závazné pokyny k~vypracování bakalářských a diplomových prací (rekapitulace pokynů, které jsou dostupné na~webu) a nezávazná doporučení od vybraných vedoucích, která již teď najdete na~webu (viz odkazy v souboru s literaturou). Jediné soubory, které se v nové verzi změní, budou \texttt{projekt-01-kapitoly-chapters.tex} a \texttt{projekt-30-prilohy-appendices.tex}, jejichž obsah každý student vymaže a nahradí vlastním. Šablonu lze tedy bez problémů využít i~v~současné verzi.

\section*{Popis částí šablony}

Po rozbalení šablony naleznete následující soubory a adresáře:
\begin{DESCRIPTION}
  \item [bib-styles] Styly literatury (viz níže). 
  \item [obrazky-figures] Adresář pro Vaše obrázky. Nyní obsahuje placeholder.pdf (tzv. TODO obrázek, který lze použít jako pomůcku při tvorbě technické zprávy), který se s prací neodevzdává. Název adresáře je vhodné zkrátit, aby byl jen ve zvoleném jazyce.
  \item [template-fig] Obrázky šablony (znak VUT).
  \item [fitthesis.cls] Šablona (definice vzhledu).
  \item [Makefile] Makefile pro překlad, počítání normostran, sbalení apod. (viz níže).
  \item [projekt-01-kapitoly-chapters.tex] Soubor pro Váš text (obsah nahraďte).
  \item [projekt-20-literatura-bibliography.bib] Seznam literatury (viz níže).
  \item [projekt-30-prilohy-appendices.tex] Soubor pro přílohy (obsah nahraďte).
  \item [projekt.tex] Hlavní soubor práce -- definice formálních částí.
\end{DESCRIPTION}

Výchozí styl literatury (czechiso) je od Ing. Martínka, přičemž anglická verze (englishiso) je jeho překladem s drobnými modifikacemi. Oproti normě jsou v něm určité odlišnosti, ale na FIT je dlouhodobě akceptován. Alternativně můžete využít styl od Ing. Radima Loskota nebo od Ing. Radka Pyšného\footnote{BP Ing. Radka Pyšného \url{http://www.fit.vutbr.cz/study/DP/BP.php?id=7848}}. Alternativní styly obsahují určitá vylepšení, ale zatím nebyly řádně otestovány větším množstvím uživatelů. Lze je považovat za beta verze pro zájemce, kteří svoji práci chtějí mít dokonalou do detailů a neváhají si nastudovat detaily správného formátování citací, aby si mohli ověřit, že je vysázený výsledek v pořádku.

Makefile kromě překladu do PDF nabízí i další funkce:
\begin{itemize}
  \item přejmenování souborů (viz níže),
  \item počítání normostran,
  \item spuštění vlny pro doplnění nezlomitelných mezer,
  \item sbalení výsledku pro odeslání vedoucímu ke kontrole (zkontrolujte, zda sbalí všechny Vámi přidané soubory, a případně doplňte).
\end{itemize}

Nezapomeňte, že vlna neřeší všechny nezlomitelné mezery. Vždy je třeba manuální kontrola, zda na konci řádku nezůstalo něco nevhodného -- viz Internetová jazyková příručka\footnote{Internetová jazyková příručka \url{http://prirucka.ujc.cas.cz/?id=880}}.

\paragraph {Pozor na číslování stránek!} Pokud má obsah 2 strany a na 2. jsou jen \uv{Přílohy} a~\uv{Seznam příloh} (ale žádná příloha tam není), z nějakého důvodu se posune číslování stránek o 1 (obsah \uv{nesedí}). Stejný efekt má, když je na 2. či 3. stránce obsahu jen \uv{Literatura} a~je možné, že tohoto problému lze dosáhnout i jinak. Řešení je několik (od~úpravy obsahu, přes nastavení počítadla až po sofistikovanější metody). \textbf{Před odevzdáním proto vždy překontrolujte číslování stran!}